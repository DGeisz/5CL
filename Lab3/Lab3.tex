
    




    
\documentclass[11pt]{article}

    
    \usepackage[breakable]{tcolorbox}
    \tcbset{nobeforeafter} % prevents tcolorboxes being placing in paragraphs
    \usepackage{float}
    \floatplacement{figure}{H} % forces figures to be placed at the correct location
    
    \usepackage[T1]{fontenc}
    % Nicer default font (+ math font) than Computer Modern for most use cases
    \usepackage{mathpazo}

    % Basic figure setup, for now with no caption control since it's done
    % automatically by Pandoc (which extracts ![](path) syntax from Markdown).
    \usepackage{graphicx}
    % We will generate all images so they have a width \maxwidth. This means
    % that they will get their normal width if they fit onto the page, but
    % are scaled down if they would overflow the margins.
    \makeatletter
    \def\maxwidth{\ifdim\Gin@nat@width>\linewidth\linewidth
    \else\Gin@nat@width\fi}
    \makeatother
    \let\Oldincludegraphics\includegraphics
    % Set max figure width to be 80% of text width, for now hardcoded.
    \renewcommand{\includegraphics}[1]{\Oldincludegraphics[width=.8\maxwidth]{#1}}
    % Ensure that by default, figures have no caption (until we provide a
    % proper Figure object with a Caption API and a way to capture that
    % in the conversion process - todo).
    \usepackage{caption}
    \DeclareCaptionLabelFormat{nolabel}{}
    \captionsetup{labelformat=nolabel}

    \usepackage{adjustbox} % Used to constrain images to a maximum size 
    \usepackage{xcolor} % Allow colors to be defined
    \usepackage{enumerate} % Needed for markdown enumerations to work
    \usepackage{geometry} % Used to adjust the document margins
    \usepackage{amsmath} % Equations
    \usepackage{amssymb} % Equations
    \usepackage{textcomp} % defines textquotesingle
    % Hack from http://tex.stackexchange.com/a/47451/13684:
    \AtBeginDocument{%
        \def\PYZsq{\textquotesingle}% Upright quotes in Pygmentized code
    }
    \usepackage{upquote} % Upright quotes for verbatim code
    \usepackage{eurosym} % defines \euro
    \usepackage[mathletters]{ucs} % Extended unicode (utf-8) support
    \usepackage[utf8x]{inputenc} % Allow utf-8 characters in the tex document
    \usepackage{fancyvrb} % verbatim replacement that allows latex
    \usepackage{grffile} % extends the file name processing of package graphics 
                         % to support a larger range 
    % The hyperref package gives us a pdf with properly built
    % internal navigation ('pdf bookmarks' for the table of contents,
    % internal cross-reference links, web links for URLs, etc.)
    \usepackage{hyperref}
    \usepackage{longtable} % longtable support required by pandoc >1.10
    \usepackage{booktabs}  % table support for pandoc > 1.12.2
    \usepackage[inline]{enumitem} % IRkernel/repr support (it uses the enumerate* environment)
    \usepackage[normalem]{ulem} % ulem is needed to support strikethroughs (\sout)
                                % normalem makes italics be italics, not underlines
    \usepackage{mathrsfs}
    

    
    % Colors for the hyperref package
    \definecolor{urlcolor}{rgb}{0,.145,.698}
    \definecolor{linkcolor}{rgb}{.71,0.21,0.01}
    \definecolor{citecolor}{rgb}{.12,.54,.11}

    % ANSI colors
    \definecolor{ansi-black}{HTML}{3E424D}
    \definecolor{ansi-black-intense}{HTML}{282C36}
    \definecolor{ansi-red}{HTML}{E75C58}
    \definecolor{ansi-red-intense}{HTML}{B22B31}
    \definecolor{ansi-green}{HTML}{00A250}
    \definecolor{ansi-green-intense}{HTML}{007427}
    \definecolor{ansi-yellow}{HTML}{DDB62B}
    \definecolor{ansi-yellow-intense}{HTML}{B27D12}
    \definecolor{ansi-blue}{HTML}{208FFB}
    \definecolor{ansi-blue-intense}{HTML}{0065CA}
    \definecolor{ansi-magenta}{HTML}{D160C4}
    \definecolor{ansi-magenta-intense}{HTML}{A03196}
    \definecolor{ansi-cyan}{HTML}{60C6C8}
    \definecolor{ansi-cyan-intense}{HTML}{258F8F}
    \definecolor{ansi-white}{HTML}{C5C1B4}
    \definecolor{ansi-white-intense}{HTML}{A1A6B2}
    \definecolor{ansi-default-inverse-fg}{HTML}{FFFFFF}
    \definecolor{ansi-default-inverse-bg}{HTML}{000000}

    % commands and environments needed by pandoc snippets
    % extracted from the output of `pandoc -s`
    \providecommand{\tightlist}{%
      \setlength{\itemsep}{0pt}\setlength{\parskip}{0pt}}
    \DefineVerbatimEnvironment{Highlighting}{Verbatim}{commandchars=\\\{\}}
    % Add ',fontsize=\small' for more characters per line
    \newenvironment{Shaded}{}{}
    \newcommand{\KeywordTok}[1]{\textcolor[rgb]{0.00,0.44,0.13}{\textbf{{#1}}}}
    \newcommand{\DataTypeTok}[1]{\textcolor[rgb]{0.56,0.13,0.00}{{#1}}}
    \newcommand{\DecValTok}[1]{\textcolor[rgb]{0.25,0.63,0.44}{{#1}}}
    \newcommand{\BaseNTok}[1]{\textcolor[rgb]{0.25,0.63,0.44}{{#1}}}
    \newcommand{\FloatTok}[1]{\textcolor[rgb]{0.25,0.63,0.44}{{#1}}}
    \newcommand{\CharTok}[1]{\textcolor[rgb]{0.25,0.44,0.63}{{#1}}}
    \newcommand{\StringTok}[1]{\textcolor[rgb]{0.25,0.44,0.63}{{#1}}}
    \newcommand{\CommentTok}[1]{\textcolor[rgb]{0.38,0.63,0.69}{\textit{{#1}}}}
    \newcommand{\OtherTok}[1]{\textcolor[rgb]{0.00,0.44,0.13}{{#1}}}
    \newcommand{\AlertTok}[1]{\textcolor[rgb]{1.00,0.00,0.00}{\textbf{{#1}}}}
    \newcommand{\FunctionTok}[1]{\textcolor[rgb]{0.02,0.16,0.49}{{#1}}}
    \newcommand{\RegionMarkerTok}[1]{{#1}}
    \newcommand{\ErrorTok}[1]{\textcolor[rgb]{1.00,0.00,0.00}{\textbf{{#1}}}}
    \newcommand{\NormalTok}[1]{{#1}}
    
    % Additional commands for more recent versions of Pandoc
    \newcommand{\ConstantTok}[1]{\textcolor[rgb]{0.53,0.00,0.00}{{#1}}}
    \newcommand{\SpecialCharTok}[1]{\textcolor[rgb]{0.25,0.44,0.63}{{#1}}}
    \newcommand{\VerbatimStringTok}[1]{\textcolor[rgb]{0.25,0.44,0.63}{{#1}}}
    \newcommand{\SpecialStringTok}[1]{\textcolor[rgb]{0.73,0.40,0.53}{{#1}}}
    \newcommand{\ImportTok}[1]{{#1}}
    \newcommand{\DocumentationTok}[1]{\textcolor[rgb]{0.73,0.13,0.13}{\textit{{#1}}}}
    \newcommand{\AnnotationTok}[1]{\textcolor[rgb]{0.38,0.63,0.69}{\textbf{\textit{{#1}}}}}
    \newcommand{\CommentVarTok}[1]{\textcolor[rgb]{0.38,0.63,0.69}{\textbf{\textit{{#1}}}}}
    \newcommand{\VariableTok}[1]{\textcolor[rgb]{0.10,0.09,0.49}{{#1}}}
    \newcommand{\ControlFlowTok}[1]{\textcolor[rgb]{0.00,0.44,0.13}{\textbf{{#1}}}}
    \newcommand{\OperatorTok}[1]{\textcolor[rgb]{0.40,0.40,0.40}{{#1}}}
    \newcommand{\BuiltInTok}[1]{{#1}}
    \newcommand{\ExtensionTok}[1]{{#1}}
    \newcommand{\PreprocessorTok}[1]{\textcolor[rgb]{0.74,0.48,0.00}{{#1}}}
    \newcommand{\AttributeTok}[1]{\textcolor[rgb]{0.49,0.56,0.16}{{#1}}}
    \newcommand{\InformationTok}[1]{\textcolor[rgb]{0.38,0.63,0.69}{\textbf{\textit{{#1}}}}}
    \newcommand{\WarningTok}[1]{\textcolor[rgb]{0.38,0.63,0.69}{\textbf{\textit{{#1}}}}}
    
    
    % Define a nice break command that doesn't care if a line doesn't already
    % exist.
    \def\br{\hspace*{\fill} \\* }
    % Math Jax compatibility definitions
    \def\gt{>}
    \def\lt{<}
    \let\Oldtex\TeX
    \let\Oldlatex\LaTeX
    \renewcommand{\TeX}{\textrm{\Oldtex}}
    \renewcommand{\LaTeX}{\textrm{\Oldlatex}}
    % Document parameters
    % Document title
    \title{	
	\normalfont\normalsize
	\textsc{ 5CL}\ % Your university, school and/or department name(s)
	\vspace{25pt} % Whitespace
	\rule{\linewidth}{0.5pt}\ % Thin top horizontal rule
	\vspace{20pt} % Whitespace
	{\huge Lab 3: Physical Optics I}\ % The assignment title
	\vspace{12pt} % Whitespace
	\rule{\linewidth}{2pt}\ % Thick bottom horizontal rule
	\vspace{12pt} % Whitespace
}

\author{\LARGE Daniel Geisz \& Heidi Hu} % Your name

\date{\normalsize\today}
    
    
    
% Pygments definitions
\makeatletter
\def\PY@reset{\let\PY@it=\relax \let\PY@bf=\relax%
    \let\PY@ul=\relax \let\PY@tc=\relax%
    \let\PY@bc=\relax \let\PY@ff=\relax}
\def\PY@tok#1{\csname PY@tok@#1\endcsname}
\def\PY@toks#1+{\ifx\relax#1\empty\else%
    \PY@tok{#1}\expandafter\PY@toks\fi}
\def\PY@do#1{\PY@bc{\PY@tc{\PY@ul{%
    \PY@it{\PY@bf{\PY@ff{#1}}}}}}}
\def\PY#1#2{\PY@reset\PY@toks#1+\relax+\PY@do{#2}}

\expandafter\def\csname PY@tok@w\endcsname{\def\PY@tc##1{\textcolor[rgb]{0.73,0.73,0.73}{##1}}}
\expandafter\def\csname PY@tok@c\endcsname{\let\PY@it=\textit\def\PY@tc##1{\textcolor[rgb]{0.25,0.50,0.50}{##1}}}
\expandafter\def\csname PY@tok@cp\endcsname{\def\PY@tc##1{\textcolor[rgb]{0.74,0.48,0.00}{##1}}}
\expandafter\def\csname PY@tok@k\endcsname{\let\PY@bf=\textbf\def\PY@tc##1{\textcolor[rgb]{0.00,0.50,0.00}{##1}}}
\expandafter\def\csname PY@tok@kp\endcsname{\def\PY@tc##1{\textcolor[rgb]{0.00,0.50,0.00}{##1}}}
\expandafter\def\csname PY@tok@kt\endcsname{\def\PY@tc##1{\textcolor[rgb]{0.69,0.00,0.25}{##1}}}
\expandafter\def\csname PY@tok@o\endcsname{\def\PY@tc##1{\textcolor[rgb]{0.40,0.40,0.40}{##1}}}
\expandafter\def\csname PY@tok@ow\endcsname{\let\PY@bf=\textbf\def\PY@tc##1{\textcolor[rgb]{0.67,0.13,1.00}{##1}}}
\expandafter\def\csname PY@tok@nb\endcsname{\def\PY@tc##1{\textcolor[rgb]{0.00,0.50,0.00}{##1}}}
\expandafter\def\csname PY@tok@nf\endcsname{\def\PY@tc##1{\textcolor[rgb]{0.00,0.00,1.00}{##1}}}
\expandafter\def\csname PY@tok@nc\endcsname{\let\PY@bf=\textbf\def\PY@tc##1{\textcolor[rgb]{0.00,0.00,1.00}{##1}}}
\expandafter\def\csname PY@tok@nn\endcsname{\let\PY@bf=\textbf\def\PY@tc##1{\textcolor[rgb]{0.00,0.00,1.00}{##1}}}
\expandafter\def\csname PY@tok@ne\endcsname{\let\PY@bf=\textbf\def\PY@tc##1{\textcolor[rgb]{0.82,0.25,0.23}{##1}}}
\expandafter\def\csname PY@tok@nv\endcsname{\def\PY@tc##1{\textcolor[rgb]{0.10,0.09,0.49}{##1}}}
\expandafter\def\csname PY@tok@no\endcsname{\def\PY@tc##1{\textcolor[rgb]{0.53,0.00,0.00}{##1}}}
\expandafter\def\csname PY@tok@nl\endcsname{\def\PY@tc##1{\textcolor[rgb]{0.63,0.63,0.00}{##1}}}
\expandafter\def\csname PY@tok@ni\endcsname{\let\PY@bf=\textbf\def\PY@tc##1{\textcolor[rgb]{0.60,0.60,0.60}{##1}}}
\expandafter\def\csname PY@tok@na\endcsname{\def\PY@tc##1{\textcolor[rgb]{0.49,0.56,0.16}{##1}}}
\expandafter\def\csname PY@tok@nt\endcsname{\let\PY@bf=\textbf\def\PY@tc##1{\textcolor[rgb]{0.00,0.50,0.00}{##1}}}
\expandafter\def\csname PY@tok@nd\endcsname{\def\PY@tc##1{\textcolor[rgb]{0.67,0.13,1.00}{##1}}}
\expandafter\def\csname PY@tok@s\endcsname{\def\PY@tc##1{\textcolor[rgb]{0.73,0.13,0.13}{##1}}}
\expandafter\def\csname PY@tok@sd\endcsname{\let\PY@it=\textit\def\PY@tc##1{\textcolor[rgb]{0.73,0.13,0.13}{##1}}}
\expandafter\def\csname PY@tok@si\endcsname{\let\PY@bf=\textbf\def\PY@tc##1{\textcolor[rgb]{0.73,0.40,0.53}{##1}}}
\expandafter\def\csname PY@tok@se\endcsname{\let\PY@bf=\textbf\def\PY@tc##1{\textcolor[rgb]{0.73,0.40,0.13}{##1}}}
\expandafter\def\csname PY@tok@sr\endcsname{\def\PY@tc##1{\textcolor[rgb]{0.73,0.40,0.53}{##1}}}
\expandafter\def\csname PY@tok@ss\endcsname{\def\PY@tc##1{\textcolor[rgb]{0.10,0.09,0.49}{##1}}}
\expandafter\def\csname PY@tok@sx\endcsname{\def\PY@tc##1{\textcolor[rgb]{0.00,0.50,0.00}{##1}}}
\expandafter\def\csname PY@tok@m\endcsname{\def\PY@tc##1{\textcolor[rgb]{0.40,0.40,0.40}{##1}}}
\expandafter\def\csname PY@tok@gh\endcsname{\let\PY@bf=\textbf\def\PY@tc##1{\textcolor[rgb]{0.00,0.00,0.50}{##1}}}
\expandafter\def\csname PY@tok@gu\endcsname{\let\PY@bf=\textbf\def\PY@tc##1{\textcolor[rgb]{0.50,0.00,0.50}{##1}}}
\expandafter\def\csname PY@tok@gd\endcsname{\def\PY@tc##1{\textcolor[rgb]{0.63,0.00,0.00}{##1}}}
\expandafter\def\csname PY@tok@gi\endcsname{\def\PY@tc##1{\textcolor[rgb]{0.00,0.63,0.00}{##1}}}
\expandafter\def\csname PY@tok@gr\endcsname{\def\PY@tc##1{\textcolor[rgb]{1.00,0.00,0.00}{##1}}}
\expandafter\def\csname PY@tok@ge\endcsname{\let\PY@it=\textit}
\expandafter\def\csname PY@tok@gs\endcsname{\let\PY@bf=\textbf}
\expandafter\def\csname PY@tok@gp\endcsname{\let\PY@bf=\textbf\def\PY@tc##1{\textcolor[rgb]{0.00,0.00,0.50}{##1}}}
\expandafter\def\csname PY@tok@go\endcsname{\def\PY@tc##1{\textcolor[rgb]{0.53,0.53,0.53}{##1}}}
\expandafter\def\csname PY@tok@gt\endcsname{\def\PY@tc##1{\textcolor[rgb]{0.00,0.27,0.87}{##1}}}
\expandafter\def\csname PY@tok@err\endcsname{\def\PY@bc##1{\setlength{\fboxsep}{0pt}\fcolorbox[rgb]{1.00,0.00,0.00}{1,1,1}{\strut ##1}}}
\expandafter\def\csname PY@tok@kc\endcsname{\let\PY@bf=\textbf\def\PY@tc##1{\textcolor[rgb]{0.00,0.50,0.00}{##1}}}
\expandafter\def\csname PY@tok@kd\endcsname{\let\PY@bf=\textbf\def\PY@tc##1{\textcolor[rgb]{0.00,0.50,0.00}{##1}}}
\expandafter\def\csname PY@tok@kn\endcsname{\let\PY@bf=\textbf\def\PY@tc##1{\textcolor[rgb]{0.00,0.50,0.00}{##1}}}
\expandafter\def\csname PY@tok@kr\endcsname{\let\PY@bf=\textbf\def\PY@tc##1{\textcolor[rgb]{0.00,0.50,0.00}{##1}}}
\expandafter\def\csname PY@tok@bp\endcsname{\def\PY@tc##1{\textcolor[rgb]{0.00,0.50,0.00}{##1}}}
\expandafter\def\csname PY@tok@fm\endcsname{\def\PY@tc##1{\textcolor[rgb]{0.00,0.00,1.00}{##1}}}
\expandafter\def\csname PY@tok@vc\endcsname{\def\PY@tc##1{\textcolor[rgb]{0.10,0.09,0.49}{##1}}}
\expandafter\def\csname PY@tok@vg\endcsname{\def\PY@tc##1{\textcolor[rgb]{0.10,0.09,0.49}{##1}}}
\expandafter\def\csname PY@tok@vi\endcsname{\def\PY@tc##1{\textcolor[rgb]{0.10,0.09,0.49}{##1}}}
\expandafter\def\csname PY@tok@vm\endcsname{\def\PY@tc##1{\textcolor[rgb]{0.10,0.09,0.49}{##1}}}
\expandafter\def\csname PY@tok@sa\endcsname{\def\PY@tc##1{\textcolor[rgb]{0.73,0.13,0.13}{##1}}}
\expandafter\def\csname PY@tok@sb\endcsname{\def\PY@tc##1{\textcolor[rgb]{0.73,0.13,0.13}{##1}}}
\expandafter\def\csname PY@tok@sc\endcsname{\def\PY@tc##1{\textcolor[rgb]{0.73,0.13,0.13}{##1}}}
\expandafter\def\csname PY@tok@dl\endcsname{\def\PY@tc##1{\textcolor[rgb]{0.73,0.13,0.13}{##1}}}
\expandafter\def\csname PY@tok@s2\endcsname{\def\PY@tc##1{\textcolor[rgb]{0.73,0.13,0.13}{##1}}}
\expandafter\def\csname PY@tok@sh\endcsname{\def\PY@tc##1{\textcolor[rgb]{0.73,0.13,0.13}{##1}}}
\expandafter\def\csname PY@tok@s1\endcsname{\def\PY@tc##1{\textcolor[rgb]{0.73,0.13,0.13}{##1}}}
\expandafter\def\csname PY@tok@mb\endcsname{\def\PY@tc##1{\textcolor[rgb]{0.40,0.40,0.40}{##1}}}
\expandafter\def\csname PY@tok@mf\endcsname{\def\PY@tc##1{\textcolor[rgb]{0.40,0.40,0.40}{##1}}}
\expandafter\def\csname PY@tok@mh\endcsname{\def\PY@tc##1{\textcolor[rgb]{0.40,0.40,0.40}{##1}}}
\expandafter\def\csname PY@tok@mi\endcsname{\def\PY@tc##1{\textcolor[rgb]{0.40,0.40,0.40}{##1}}}
\expandafter\def\csname PY@tok@il\endcsname{\def\PY@tc##1{\textcolor[rgb]{0.40,0.40,0.40}{##1}}}
\expandafter\def\csname PY@tok@mo\endcsname{\def\PY@tc##1{\textcolor[rgb]{0.40,0.40,0.40}{##1}}}
\expandafter\def\csname PY@tok@ch\endcsname{\let\PY@it=\textit\def\PY@tc##1{\textcolor[rgb]{0.25,0.50,0.50}{##1}}}
\expandafter\def\csname PY@tok@cm\endcsname{\let\PY@it=\textit\def\PY@tc##1{\textcolor[rgb]{0.25,0.50,0.50}{##1}}}
\expandafter\def\csname PY@tok@cpf\endcsname{\let\PY@it=\textit\def\PY@tc##1{\textcolor[rgb]{0.25,0.50,0.50}{##1}}}
\expandafter\def\csname PY@tok@c1\endcsname{\let\PY@it=\textit\def\PY@tc##1{\textcolor[rgb]{0.25,0.50,0.50}{##1}}}
\expandafter\def\csname PY@tok@cs\endcsname{\let\PY@it=\textit\def\PY@tc##1{\textcolor[rgb]{0.25,0.50,0.50}{##1}}}

\def\PYZbs{\char`\\}
\def\PYZus{\char`\_}
\def\PYZob{\char`\{}
\def\PYZcb{\char`\}}
\def\PYZca{\char`\^}
\def\PYZam{\char`\&}
\def\PYZlt{\char`\<}
\def\PYZgt{\char`\>}
\def\PYZsh{\char`\#}
\def\PYZpc{\char`\%}
\def\PYZdl{\char`\$}
\def\PYZhy{\char`\-}
\def\PYZsq{\char`\'}
\def\PYZdq{\char`\"}
\def\PYZti{\char`\~}
% for compatibility with earlier versions
\def\PYZat{@}
\def\PYZlb{[}
\def\PYZrb{]}
\makeatother


    % For linebreaks inside Verbatim environment from package fancyvrb. 
    \makeatletter
        \newbox\Wrappedcontinuationbox 
        \newbox\Wrappedvisiblespacebox 
        \newcommand*\Wrappedvisiblespace {\textcolor{red}{\textvisiblespace}} 
        \newcommand*\Wrappedcontinuationsymbol {\textcolor{red}{\llap{\tiny$\m@th\hookrightarrow$}}} 
        \newcommand*\Wrappedcontinuationindent {3ex } 
        \newcommand*\Wrappedafterbreak {\kern\Wrappedcontinuationindent\copy\Wrappedcontinuationbox} 
        % Take advantage of the already applied Pygments mark-up to insert 
        % potential linebreaks for TeX processing. 
        %        {, <, #, %, $, ' and ": go to next line. 
        %        _, }, ^, &, >, - and ~: stay at end of broken line. 
        % Use of \textquotesingle for straight quote. 
        \newcommand*\Wrappedbreaksatspecials {% 
            \def\PYGZus{\discretionary{\char`\_}{\Wrappedafterbreak}{\char`\_}}% 
            \def\PYGZob{\discretionary{}{\Wrappedafterbreak\char`\{}{\char`\{}}% 
            \def\PYGZcb{\discretionary{\char`\}}{\Wrappedafterbreak}{\char`\}}}% 
            \def\PYGZca{\discretionary{\char`\^}{\Wrappedafterbreak}{\char`\^}}% 
            \def\PYGZam{\discretionary{\char`\&}{\Wrappedafterbreak}{\char`\&}}% 
            \def\PYGZlt{\discretionary{}{\Wrappedafterbreak\char`\<}{\char`\<}}% 
            \def\PYGZgt{\discretionary{\char`\>}{\Wrappedafterbreak}{\char`\>}}% 
            \def\PYGZsh{\discretionary{}{\Wrappedafterbreak\char`\#}{\char`\#}}% 
            \def\PYGZpc{\discretionary{}{\Wrappedafterbreak\char`\%}{\char`\%}}% 
            \def\PYGZdl{\discretionary{}{\Wrappedafterbreak\char`\$}{\char`\$}}% 
            \def\PYGZhy{\discretionary{\char`\-}{\Wrappedafterbreak}{\char`\-}}% 
            \def\PYGZsq{\discretionary{}{\Wrappedafterbreak\textquotesingle}{\textquotesingle}}% 
            \def\PYGZdq{\discretionary{}{\Wrappedafterbreak\char`\"}{\char`\"}}% 
            \def\PYGZti{\discretionary{\char`\~}{\Wrappedafterbreak}{\char`\~}}% 
        } 
        % Some characters . , ; ? ! / are not pygmentized. 
        % This macro makes them "active" and they will insert potential linebreaks 
        \newcommand*\Wrappedbreaksatpunct {% 
            \lccode`\~`\.\lowercase{\def~}{\discretionary{\hbox{\char`\.}}{\Wrappedafterbreak}{\hbox{\char`\.}}}% 
            \lccode`\~`\,\lowercase{\def~}{\discretionary{\hbox{\char`\,}}{\Wrappedafterbreak}{\hbox{\char`\,}}}% 
            \lccode`\~`\;\lowercase{\def~}{\discretionary{\hbox{\char`\;}}{\Wrappedafterbreak}{\hbox{\char`\;}}}% 
            \lccode`\~`\:\lowercase{\def~}{\discretionary{\hbox{\char`\:}}{\Wrappedafterbreak}{\hbox{\char`\:}}}% 
            \lccode`\~`\?\lowercase{\def~}{\discretionary{\hbox{\char`\?}}{\Wrappedafterbreak}{\hbox{\char`\?}}}% 
            \lccode`\~`\!\lowercase{\def~}{\discretionary{\hbox{\char`\!}}{\Wrappedafterbreak}{\hbox{\char`\!}}}% 
            \lccode`\~`\/\lowercase{\def~}{\discretionary{\hbox{\char`\/}}{\Wrappedafterbreak}{\hbox{\char`\/}}}% 
            \catcode`\.\active
            \catcode`\,\active 
            \catcode`\;\active
            \catcode`\:\active
            \catcode`\?\active
            \catcode`\!\active
            \catcode`\/\active 
            \lccode`\~`\~ 	
        }
    \makeatother

    \let\OriginalVerbatim=\Verbatim
    \makeatletter
    \renewcommand{\Verbatim}[1][1]{%
        %\parskip\z@skip
        \sbox\Wrappedcontinuationbox {\Wrappedcontinuationsymbol}%
        \sbox\Wrappedvisiblespacebox {\FV@SetupFont\Wrappedvisiblespace}%
        \def\FancyVerbFormatLine ##1{\hsize\linewidth
            \vtop{\raggedright\hyphenpenalty\z@\exhyphenpenalty\z@
                \doublehyphendemerits\z@\finalhyphendemerits\z@
                \strut ##1\strut}%
        }%
        % If the linebreak is at a space, the latter will be displayed as visible
        % space at end of first line, and a continuation symbol starts next line.
        % Stretch/shrink are however usually zero for typewriter font.
        \def\FV@Space {%
            \nobreak\hskip\z@ plus\fontdimen3\font minus\fontdimen4\font
            \discretionary{\copy\Wrappedvisiblespacebox}{\Wrappedafterbreak}
            {\kern\fontdimen2\font}%
        }%
        
        % Allow breaks at special characters using \PYG... macros.
        \Wrappedbreaksatspecials
        % Breaks at punctuation characters . , ; ? ! and / need catcode=\active 	
        \OriginalVerbatim[#1,codes*=\Wrappedbreaksatpunct]%
    }
    \makeatother

    % Exact colors from NB
    \definecolor{incolor}{HTML}{303F9F}
    \definecolor{outcolor}{HTML}{D84315}
    \definecolor{cellborder}{HTML}{CFCFCF}
    \definecolor{cellbackground}{HTML}{F7F7F7}
    
    % prompt
    \newcommand{\prompt}[4]{
        \llap{{\color{#2}[#3]: #4}}\vspace{-1.25em}
    }
    

    
    % Prevent overflowing lines due to hard-to-break entities
    \sloppy 
    % Setup hyperref package
    \hypersetup{
      breaklinks=true,  % so long urls are correctly broken across lines
      colorlinks=true,
      urlcolor=urlcolor,
      linkcolor=linkcolor,
      citecolor=citecolor,
      }
    % Slightly bigger margins than the latex defaults
    
    \geometry{verbose,tmargin=1in,bmargin=1in,lmargin=1in,rmargin=1in}
    
    

    \begin{document}
    
    
    \maketitle

    \hypertarget{experiment-2---varying-the-single-slit-diffraction-pattern}{%
\subsection{Experiment 2 - Varying the Single Slit Diffraction
Pattern}\label{experiment-2---varying-the-single-slit-diffraction-pattern}}

\hypertarget{objective}{%
\paragraph{Objective:}\label{objective}}

\begin{itemize}
\tightlist
\item
  In this experiment, we looking at different ways that we can alter the
  single-slit diffraction pattern that shows up on the screen. In this
  lab, we change the slit width, the wavelength of the light
  \(\lambda\), and we use a diverging lens to learn how different
  factors change the diffraction pattern.
\end{itemize}

\hypertarget{method}{%
\paragraph{Method:}\label{method}}

\begin{itemize}
\item
  Set up materials according to the figure in section 1.1 of the lab
  manual.
\item
  In this lab, the main equations we were testing were:
\end{itemize}

\[ \Delta y_{cent} = \frac{2L\lambda}{a}\]

\[ \Delta y_{side} = \frac{L\lambda}{a}\]

\begin{itemize}
\item
  In order to determine the effect changing the slit width has on the
  pattern on the screen, we simply observed the diffraction pattern
  resulting from different slit widths on the grating plate. Based on
  the above equations, we predicted that \(y_{cent}\) should decrease
  when we increased slit length.
\item
  In order to test the effect changing the wavelength of the light
  \(\lambda\) had on the diffraction pattern, we used a green laser in
  addition to the red laser, and noted the differences between the
  resultant diffraction patterns. Based on the above equation, we
  predicted that \(y_{cent}\) would decrease when we decreased
  wavelength.
\item
  Finally, to test the effect of the diverging lens on the diffraction
  pattern, we placed the diverging lens in front of the grating plate,
  and then behind the grating plate and observed the changes each
  placement of the grating plate had on the resultant diffraction
  pattern.
\end{itemize}

    \hypertarget{answers-to-data-analysis}{%
\paragraph{Answers to Data Analysis:}\label{answers-to-data-analysis}}

\begin{enumerate}
\def\labelenumi{\alph{enumi})}
\tightlist
\item
  Based on your your observations and the theoretical dependence of the
  diffraction pattern width on slit width, determine if any of your data
  points significantly deviates from what you would expect given the
  listed slit widths. Discuss whether or not it would be appropriate to
  ``ignore'' this data point in our data analysis.
\end{enumerate}

Answer:

The following data table represents the data we took for
\(y_1\)-vs-\(a\). Here ``a'' represents the slit width measured in
points, ``y1'' represents the location of the first minimum in
centimeters, and ``delta\_y1'' represents the uncertainty in the
measurement of ``y1''.

    \begin{tcolorbox}[breakable, size=fbox, boxrule=1pt, pad at break*=1mm,colback=cellbackground, colframe=cellborder]
\prompt{In}{incolor}{5}{\hspace{4pt}}
\begin{Verbatim}[commandchars=\\\{\}]
\PY{n}{exp3\PYZus{}data}
\end{Verbatim}
\end{tcolorbox}

            \begin{tcolorbox}[breakable, boxrule=.5pt, size=fbox, pad at break*=1mm, opacityfill=0]
\prompt{Out}{outcolor}{5}{\hspace{3.5pt}}
\begin{Verbatim}[commandchars=\\\{\}]
    a     y1  delta\_y1
0   2  0.350  0.000625
1   4  0.209  0.000625
2   8  0.130  0.000410
3  16  0.060  0.000500
\end{Verbatim}
\end{tcolorbox}
        
    We also measured the distance between the grating plate and the screen
to be \(67.4 \pm 0.05\) cm.

    In order to determine whether any data points significantly deviate from
the expected value, I'll use the our data to calculate slit width and
compare it to the listed value. The following script performs this
calculation. ``a'' is the same value as in the previous data table, and
``Listed Value'' is just the value of ``a'' from the previous data table
times 0.004393 mm to find the listed slit width in millimeters. ``Actual
Value'' and ``Actual Value Uncertainty'' also both have units of
millimeters.

    \begin{tcolorbox}[breakable, size=fbox, boxrule=1pt, pad at break*=1mm,colback=cellbackground, colframe=cellborder]
\prompt{In}{incolor}{9}{\hspace{4pt}}
\begin{Verbatim}[commandchars=\\\{\}]
\PY{n}{compare} \PY{o}{=} \PY{n}{pd}\PY{o}{.}\PY{n}{DataFrame}\PY{p}{(}\PY{p}{)}
\PY{n}{ex} \PY{o}{=} \PY{n}{exp3\PYZus{}data}
\PY{n}{compare}\PY{p}{[}\PY{l+s+s1}{\PYZsq{}}\PY{l+s+s1}{a}\PY{l+s+s1}{\PYZsq{}}\PY{p}{]} \PY{o}{=} \PY{n}{exp3\PYZus{}data}\PY{o}{.}\PY{n}{a}
\PY{n}{compare}\PY{p}{[}\PY{l+s+s1}{\PYZsq{}}\PY{l+s+s1}{Listed Value}\PY{l+s+s1}{\PYZsq{}}\PY{p}{]} \PY{o}{=} \PY{n}{exp3\PYZus{}data}\PY{o}{.}\PY{n}{a} \PY{o}{*} \PY{l+m+mf}{0.04393}
\PY{n}{compare}\PY{p}{[}\PY{l+s+s1}{\PYZsq{}}\PY{l+s+s1}{Actual Value}\PY{l+s+s1}{\PYZsq{}}\PY{p}{]} \PY{o}{=} \PY{l+m+mi}{674} \PY{o}{*} \PY{p}{(}\PY{l+m+mf}{6.346e\PYZhy{}5}\PY{p}{)} \PY{o}{/} \PY{n}{exp3\PYZus{}data}\PY{o}{.}\PY{n}{y1}
\PY{n}{compare}\PY{p}{[}\PY{l+s+s1}{\PYZsq{}}\PY{l+s+s1}{Actual Value Uncertainty}\PY{l+s+s1}{\PYZsq{}}\PY{p}{]} \PY{o}{=} \PY{n}{compare}\PY{p}{[}\PY{l+s+s1}{\PYZsq{}}\PY{l+s+s1}{Actual Value}\PY{l+s+s1}{\PYZsq{}}\PY{p}{]} \PY{o}{*} \PY{p}{(}\PY{p}{(}\PY{n}{ex}\PY{o}{.}\PY{n}{delta\PYZus{}y1} \PY{o}{/} \PY{n}{ex}\PY{o}{.}\PY{n}{y1}\PY{p}{)}\PY{o}{*}\PY{o}{*}\PY{l+m+mi}{2} \PY{o}{+} \PY{p}{(}\PY{l+m+mf}{0.05} \PY{o}{/} \PY{l+m+mf}{67.4}\PY{p}{)}\PY{o}{*}\PY{o}{*}\PY{l+m+mi}{2}\PY{p}{)}\PY{o}{*}\PY{o}{*}\PY{l+m+mf}{0.5}
\PY{n}{compare}
\end{Verbatim}
\end{tcolorbox}

            \begin{tcolorbox}[breakable, boxrule=.5pt, size=fbox, pad at break*=1mm, opacityfill=0]
\prompt{Out}{outcolor}{9}{\hspace{3.5pt}}
\begin{Verbatim}[commandchars=\\\{\}]
    a  Listed Value  Actual Value  Actual Value Uncertainty
0   2       0.08786      0.122206                  0.000236
1   4       0.17572      0.204651                  0.000631
2   8       0.35144      0.329016                  0.001066
3  16       0.70288      0.712867                  0.005964
\end{Verbatim}
\end{tcolorbox}
        
    This final data plots the difference between the Listed and Actual
Values of slit width from the previous table:

    \begin{tcolorbox}[breakable, size=fbox, boxrule=1pt, pad at break*=1mm,colback=cellbackground, colframe=cellborder]
\prompt{In}{incolor}{15}{\hspace{4pt}}
\begin{Verbatim}[commandchars=\\\{\}]
\PY{n}{final} \PY{o}{=} \PY{n}{pd}\PY{o}{.}\PY{n}{DataFrame}\PY{p}{(}\PY{p}{)}
\PY{n}{final}\PY{p}{[}\PY{l+s+s1}{\PYZsq{}}\PY{l+s+s1}{a}\PY{l+s+s1}{\PYZsq{}}\PY{p}{]} \PY{o}{=} \PY{n}{ex}\PY{o}{.}\PY{n}{a}
\PY{n}{final}\PY{p}{[}\PY{l+s+s1}{\PYZsq{}}\PY{l+s+s1}{Percent Difference}\PY{l+s+s1}{\PYZsq{}}\PY{p}{]} \PY{o}{=} \PY{l+m+mi}{100} \PY{o}{*} \PY{n+nb}{abs}\PY{p}{(}\PY{n}{compare}\PY{p}{[}\PY{l+s+s1}{\PYZsq{}}\PY{l+s+s1}{Actual Value}\PY{l+s+s1}{\PYZsq{}}\PY{p}{]} \PY{o}{\PYZhy{}} \PY{n}{compare}\PY{p}{[}\PY{l+s+s1}{\PYZsq{}}\PY{l+s+s1}{Listed Value}\PY{l+s+s1}{\PYZsq{}}\PY{p}{]}\PY{p}{)} \PY{o}{/} \PY{n}{compare}\PY{p}{[}\PY{l+s+s1}{\PYZsq{}}\PY{l+s+s1}{Actual Value}\PY{l+s+s1}{\PYZsq{}}\PY{p}{]}
\PY{n}{final}
\end{Verbatim}
\end{tcolorbox}

\pagebreak
            \begin{tcolorbox}[breakable, boxrule=.5pt, size=fbox, pad at break*=1mm, opacityfill=0]
\prompt{Out}{outcolor}{15}{\hspace{3.5pt}}
\begin{Verbatim}[commandchars=\\\{\}]
    a  Percent Difference
0   2           28.104902
1   4           14.136712
2   8            6.815574
3  16            1.401009
\end{Verbatim}
\end{tcolorbox}
        
    We therefore see that the \(a=2\) slit has the largest discrepancy
between the listed and actual slit width. It would be appropriate to
drop this data point simply because the actual difference between the
actual and listed value of slit width is about 0.04 mm, which is an
extremely small length based on the size of the equipment used to
manufacture the grating plate. Also the fact that the listed value is
28\% smaller than our experimentally determined value means it might
have a detrimental effect on the rest of our data. However, because we
have so few data points already, we will keep this data point in the
following analysis.

    \begin{enumerate}
\def\labelenumi{\alph{enumi})}
\setcounter{enumi}{1}
\tightlist
\item
  Define the new quantities z ≡ ln(y1/mm) and w ≡ ln(a/pt) and make a
  data table and scatter-plot of z-vs-w with error bars for the
  uncertainty in z.
\end{enumerate}

The following table shows the calculated values of ``w'' and ``z'', and
it also includes the uncertainty in ``z'', denoted ``dz''.

    \begin{tcolorbox}[breakable, size=fbox, boxrule=1pt, pad at break*=1mm,colback=cellbackground, colframe=cellborder]
\prompt{In}{incolor}{33}{\hspace{4pt}}
\begin{Verbatim}[commandchars=\\\{\}]
\PY{n}{wz}
\end{Verbatim}
\end{tcolorbox}

            \begin{tcolorbox}[breakable, boxrule=.5pt, size=fbox, pad at break*=1mm, opacityfill=0]
\prompt{Out}{outcolor}{33}{\hspace{3.5pt}}
\begin{Verbatim}[commandchars=\\\{\}]
          w         z        dz
0  0.693147 -1.049822 -0.000595
1  1.386294 -1.565421 -0.000399
2  2.079442 -2.040221 -0.000201
3  2.772589 -2.813411 -0.000178
\end{Verbatim}
\end{tcolorbox}
        
    \begin{tcolorbox}[breakable, size=fbox, boxrule=1pt, pad at break*=1mm,colback=cellbackground, colframe=cellborder]
\prompt{In}{incolor}{46}{\hspace{4pt}}
\begin{Verbatim}[commandchars=\\\{\}]
\PY{n}{plt}\PY{o}{.}\PY{n}{errorbar}\PY{p}{(}\PY{n}{wz}\PY{o}{.}\PY{n}{w}\PY{p}{,} \PY{n}{wz}\PY{o}{.}\PY{n}{z}\PY{p}{,} \PY{n}{yerr} \PY{o}{=} \PY{n}{wz}\PY{o}{.}\PY{n}{dz}\PY{p}{)}
\PY{n}{plt}\PY{o}{.}\PY{n}{title}\PY{p}{(}\PY{l+s+s1}{\PYZsq{}}\PY{l+s+s1}{W vs Z}\PY{l+s+s1}{\PYZsq{}}\PY{p}{)}
\PY{n}{plt}\PY{o}{.}\PY{n}{xlabel}\PY{p}{(}\PY{l+s+s1}{\PYZsq{}}\PY{l+s+s1}{w}\PY{l+s+s1}{\PYZsq{}}\PY{p}{)}
\PY{n}{plt}\PY{o}{.}\PY{n}{ylabel}\PY{p}{(}\PY{l+s+s1}{\PYZsq{}}\PY{l+s+s1}{z}\PY{l+s+s1}{\PYZsq{}}\PY{p}{)}
\PY{n}{plt}\PY{o}{.}\PY{n}{show}\PY{p}{(}\PY{p}{)}
\end{Verbatim}
\end{tcolorbox}

    \begin{center}
    \adjustimage{max size={0.9\linewidth}{0.9\paperheight}}{yote.png}
    \end{center}
    { \hspace*{\fill} \\}

\textbf{Note: } The error bars are difficult to see because they are so small.
    
    \begin{enumerate}
\def\labelenumi{\alph{enumi})}
\setcounter{enumi}{2}
\tightlist
\item
  Determine the theoretically expected values for the slope m and
  intercept b.
\end{enumerate}

Based on the equation:

\[ y_1 = \frac{L\lambda}{a}\]

We see,

\[ \ln(y_1) = \ln(L\lambda) - \ln(a)\]
\[ \implies z = -w + \ln(L\lambda)\]

So then, we theoretically expect that:

\[\boxed{m = -1 \text{,   } b = \ln(L\lambda) = \ln(674 \cdot (6.346 \cdot 10 ^{-5})) = -3.15}\]

    \begin{enumerate}
\def\labelenumi{\alph{enumi})}
\setcounter{enumi}{2}
\tightlist
\item
  Use a least-squares fit to determine the fit values (with
  uncertainties) for your data.
\end{enumerate}

The following script uses least squares to calculate \(m\), \(b\) and
their corresponding uncertainties.

    \begin{tcolorbox}[breakable, size=fbox, boxrule=1pt, pad at break*=1mm,colback=cellbackground, colframe=cellborder]
\prompt{In}{incolor}{41}{\hspace{4pt}}
\begin{Verbatim}[commandchars=\\\{\}]
\PY{n}{denom} \PY{o}{=} \PY{p}{(}\PY{n}{wz}\PY{o}{.}\PY{n}{w} \PY{o}{*}\PY{o}{*} \PY{l+m+mi}{2}\PY{p}{)}\PY{o}{.}\PY{n}{mean}\PY{p}{(}\PY{p}{)} \PY{o}{\PYZhy{}} \PY{p}{(}\PY{n}{wz}\PY{o}{.}\PY{n}{w}\PY{o}{.}\PY{n}{mean}\PY{p}{(}\PY{p}{)} \PY{o}{*}\PY{o}{*}\PY{l+m+mi}{2}\PY{p}{)}
\PY{n}{m}\PY{o}{=} \PY{p}{(}\PY{p}{(}\PY{n}{wz}\PY{o}{.}\PY{n}{z} \PY{o}{*} \PY{n}{wz}\PY{o}{.}\PY{n}{w}\PY{p}{)}\PY{o}{.}\PY{n}{mean}\PY{p}{(}\PY{p}{)} \PY{o}{\PYZhy{}} \PY{p}{(}\PY{n}{wz}\PY{o}{.}\PY{n}{z}\PY{o}{.}\PY{n}{mean}\PY{p}{(}\PY{p}{)} \PY{o}{*} \PY{n}{wz}\PY{o}{.}\PY{n}{w}\PY{o}{.}\PY{n}{mean}\PY{p}{(}\PY{p}{)}\PY{p}{)}\PY{p}{)}  \PY{o}{/} \PY{n}{denom}
\PY{n}{b} \PY{o}{=} \PY{n}{wz}\PY{o}{.}\PY{n}{z}\PY{o}{.}\PY{n}{mean}\PY{p}{(}\PY{p}{)} \PY{o}{\PYZhy{}} \PY{p}{(}\PY{n}{m} \PY{o}{*} \PY{n}{wz}\PY{o}{.}\PY{n}{w}\PY{o}{.}\PY{n}{mean}\PY{p}{(}\PY{p}{)}\PY{p}{)}
\PY{n}{dy} \PY{o}{=} \PY{p}{(}\PY{l+m+mf}{0.5}\PY{o}{*}\PY{p}{(}\PY{p}{(}\PY{n}{wz}\PY{o}{.}\PY{n}{z} \PY{o}{\PYZhy{}} \PY{p}{(}\PY{n}{wz}\PY{o}{.}\PY{n}{w} \PY{o}{*} \PY{n}{m} \PY{o}{+} \PY{n}{b}\PY{p}{)}\PY{p}{)}\PY{o}{*}\PY{o}{*}\PY{l+m+mi}{2}\PY{p}{)}\PY{o}{.}\PY{n}{sum}\PY{p}{(}\PY{p}{)}\PY{p}{)}\PY{o}{*}\PY{o}{*}\PY{l+m+mf}{0.5}
\PY{n}{dm} \PY{o}{=} \PY{n}{dy} \PY{o}{/} \PY{n}{sqrt}\PY{p}{(}\PY{l+m+mi}{4} \PY{o}{*} \PY{n}{denom}\PY{p}{)}
\PY{n}{db} \PY{o}{=} \PY{n}{sqrt}\PY{p}{(}\PY{p}{(}\PY{n}{wz}\PY{o}{.}\PY{n}{w}\PY{o}{*}\PY{o}{*}\PY{l+m+mi}{2}\PY{p}{)}\PY{o}{.}\PY{n}{mean}\PY{p}{(}\PY{p}{)}\PY{p}{)} \PY{o}{*} \PY{n}{dm}

\PY{n+nb}{print}\PY{p}{(}\PY{l+s+s2}{\PYZdq{}}\PY{l+s+s2}{m: }\PY{l+s+s2}{\PYZdq{}}\PY{p}{,} \PY{n}{m}\PY{p}{,} \PY{l+s+s2}{\PYZdq{}}\PY{l+s+s2}{ +\PYZhy{} }\PY{l+s+s2}{\PYZdq{}}\PY{p}{,} \PY{n}{dm}\PY{p}{)}
\PY{n+nb}{print}\PY{p}{(}\PY{l+s+s2}{\PYZdq{}}\PY{l+s+s2}{b: }\PY{l+s+s2}{\PYZdq{}}\PY{p}{,} \PY{n}{b}\PY{p}{,} \PY{l+s+s2}{\PYZdq{}}\PY{l+s+s2}{ +\PYZhy{}}\PY{l+s+s2}{\PYZdq{}}\PY{p}{,} \PY{n}{db}\PY{p}{)}
\end{Verbatim}
\end{tcolorbox}

    \begin{Verbatim}[commandchars=\\\{\}]
m:  -0.8317952867723865  +-  0.0681904187269391
b:  -0.4258272796273217  +- 0.1294432623801534
\end{Verbatim}

    Therefore the least-squares fit yields best fit values of:

\[\boxed{m = -0.832 \pm 0.068 \text{, } b = -0.426 \pm 0.129}\]

    \hypertarget{experiment-4}{%
\subsection{Experiment 4}\label{experiment-4}}

\hypertarget{objective}{%
\paragraph{Objective}\label{objective}}

In this experiment, we decided to test the equation that states when
Intensity Maxima occur for multiple thin-slit interference. This
equation is given by:

\[d\sin(\theta) = m\lambda\text{, }m\in \mathbb{Z}\]

Specifically, we wished to show that the locations of the bright maxima
should not change when we varied N, the number of slits.

\hypertarget{method}{%
\paragraph{Method}\label{method}}

First we set up the materials according to the figure in section 1.1 of
the Lab Manual. We used the method from earlier labs to align the laser.
Then we adjusted the light so that it would go through the 3/1/2 slit
pattern. Finally we measured the distance between the center of the
pattern to the first bright maximum. \\

We then repeated this procedure for the 4/1/2, 10/1/2, and 20/1/2 slits.

\hypertarget{data-analysis}{%
\paragraph{Data Analysis}\label{data-analysis}}

The following table shows the data we collected for this experiment. In
this table, `n' stands for the number of slits used; `a' stands for the
thickness of slits in pts; `d' stands for the widths of slits in pts;
`w' and `dw' stand for the location and uncertainty of the first bright
maxima in millimeters.

    \begin{tcolorbox}[breakable, size=fbox, boxrule=1pt, pad at break*=1mm,colback=cellbackground, colframe=cellborder]
\prompt{In}{incolor}{26}{\hspace{4pt}}
\begin{Verbatim}[commandchars=\\\{\}]
\PY{n}{df}
\end{Verbatim}
\end{tcolorbox}

            \begin{tcolorbox}[breakable, boxrule=.5pt, size=fbox, pad at break*=1mm, opacityfill=0]
\prompt{Out}{outcolor}{26}{\hspace{3.5pt}}
\begin{Verbatim}[commandchars=\\\{\}]
    n  a  d     w     dw
0   3  1  2  3.46  0.005
1   4  1  2  3.48  0.005
2  10  1  2  3.51  0.005
3  20  1  2  3.52  0.005
\end{Verbatim}
\end{tcolorbox}
        
    Because we are specifically trying to show that there is no dependence
between the number of slits and the location of the first bright
maximum, we fitted our data to a linear hypothesis
\(y=\hat{m}x+\hat{b}\) with the intention of showing \(\hat{m} = 0\) and
$\hat{b} \approx 3.5 $ Using scripts found in appendix, we found that:

\[\hat{m}=0.003\pm0.0009 \text{, } \hat{b}=3.463\pm0.0104\]

Clearly these values of \(\hat{m}\) and \(\hat{b}\) match our
expectations with small error. In order to determine if this hypothesis
was a good fit of the data, we used a residuals test. The following
script plots the residuals:

    \begin{tcolorbox}[breakable, size=fbox, boxrule=1pt, pad at break*=1mm,colback=cellbackground, colframe=cellborder]
\prompt{In}{incolor}{29}{\hspace{4pt}}
\begin{Verbatim}[commandchars=\\\{\}]
\PY{n}{plt}\PY{o}{.}\PY{n}{errorbar}\PY{p}{(}\PY{n}{df}\PY{o}{.}\PY{n}{n}\PY{p}{,} \PY{n}{df}\PY{o}{.}\PY{n}{residual}\PY{p}{,} \PY{n}{yerr} \PY{o}{=} \PY{n}{df}\PY{o}{.}\PY{n}{delresid}\PY{p}{,} \PY{n}{fmt} \PY{o}{=} \PY{l+s+s1}{\PYZsq{}}\PY{l+s+s1}{.}\PY{l+s+s1}{\PYZsq{}}\PY{p}{)}
\PY{n}{plt}\PY{o}{.}\PY{n}{xlabel}\PY{p}{(}\PY{l+s+s1}{\PYZsq{}}\PY{l+s+s1}{slit number}\PY{l+s+s1}{\PYZsq{}}\PY{p}{)}
\PY{n}{plt}\PY{o}{.}\PY{n}{ylabel}\PY{p}{(}\PY{l+s+s1}{\PYZsq{}}\PY{l+s+s1}{residual}\PY{l+s+s1}{\PYZsq{}}\PY{p}{)}
\PY{n}{plt}\PY{o}{.}\PY{n}{title}\PY{p}{(}\PY{l+s+s1}{\PYZsq{}}\PY{l+s+s1}{residual scatter plot}\PY{l+s+s1}{\PYZsq{}}\PY{p}{)}
\PY{n}{plt}\PY{o}{.}\PY{n}{axhline}\PY{p}{(}\PY{l+m+mi}{0}\PY{p}{,} \PY{l+m+mi}{0}\PY{p}{,} \PY{l+m+mi}{1}\PY{p}{,} \PY{n}{color} \PY{o}{=} \PY{l+s+s1}{\PYZsq{}}\PY{l+s+s1}{red}\PY{l+s+s1}{\PYZsq{}}\PY{p}{)}
\PY{n}{plt}\PY{o}{.}\PY{n}{show}\PY{p}{(}\PY{p}{)}
\end{Verbatim}
\end{tcolorbox}

    \begin{center}
    \adjustimage{max size={0.9\linewidth}{0.9\paperheight}}{output_22_0.png}
    \end{center}
    { \hspace*{\fill} \\}
    
    Clearly more than 2/3 of the error bars cross the x-axis, and in fact
all the error bars cross the x-axis, so we can say that this hypothesis
is a very good fit of the data. Therefore our data analysis strongly
supports the theoretical prediction that the location of the first
bright maximum should not depend on the number of slits in the grating.

    \hypertarget{summary-and-conclusions}{%
\subsection{Summary and Conclusions}\label{summary-and-conclusions}}

The purpose of this lab was to explore the qualitative features of the
interference and diffraction patterns formed by slits, and to Use the
quantitative features of interference and diffraction patterns to
measure small lengths. The very nature of this lab allowed us to explore
different features of interference and diffraction patterns, and we used
measurements of the location of a pattern's maxima together with the
distances between the different pieces of equipment to measure the slit
width, which is indeed a small quantity. Therefore we fulfilled the
purpose of this lab.

More specifically, through this lab we were able to test the various
equations governing the widths of maxima in interference and diffraction
patterns, and we were able to verify that these equations hold for
various setups.

Finally, to reiterate the findings of the two experiments listed in this
report, in the first experiment we showed that one of the slit widths
had a large discrepancy between the listed value and the experimentally
determined value. We also showed that we could take a natural log of the
equation governing the system in order to fit our data to linear
hypothesis. In the fourth experiment, we showed that the location of the
first bright maximum of a multiple-slit interference pattern does not
depend of the number of slits in the grating.

\hypertarget{contributions-of-each-member}{%
\subsection{Contributions of Each
Member}\label{contributions-of-each-member}}

Danny and Heidi each did 50\% of the lab work and the data analysis.
Danny did 70\% of the work writing out the actual findings of the data
analysis.

\hypertarget{references}{%
\subsection{References}\label{references}}

We used python, and the open source python libraries MatPlotLib, Pandas,
and NumPy.

    \begin{tcolorbox}[breakable, size=fbox, boxrule=1pt, pad at break*=1mm,colback=cellbackground, colframe=cellborder]
\prompt{In}{incolor}{16}{\hspace{4pt}}
\begin{Verbatim}[commandchars=\\\{\}]
\PY{k+kn}{import} \PY{n+nn}{pandas} \PY{k}{as} \PY{n+nn}{pd}
\PY{k+kn}{import} \PY{n+nn}{numpy} \PY{k}{as} \PY{n+nn}{np}
\PY{k+kn}{import} \PY{n+nn}{matplotlib}\PY{n+nn}{.}\PY{n+nn}{pyplot} \PY{k}{as} \PY{n+nn}{plt}
\PY{k+kn}{from} \PY{n+nn}{math} \PY{k}{import} \PY{o}{*}
\end{Verbatim}
\end{tcolorbox}

    The following scripts were used to do the linear fitting in Experiment
4:

    \begin{tcolorbox}[breakable, size=fbox, boxrule=1pt, pad at break*=1mm,colback=cellbackground, colframe=cellborder]
\prompt{In}{incolor}{4}{\hspace{4pt}}
\begin{Verbatim}[commandchars=\\\{\}]
\PY{n}{data\PYZus{}dict} \PY{o}{=} \PY{p}{\PYZob{}}\PY{l+s+s2}{\PYZdq{}}\PY{l+s+s2}{a}\PY{l+s+s2}{\PYZdq{}} \PY{p}{:} \PY{p}{[}\PY{l+m+mi}{2}\PY{p}{,} \PY{l+m+mi}{4}\PY{p}{,} \PY{l+m+mi}{8}\PY{p}{,} \PY{l+m+mi}{16}\PY{p}{]}\PY{p}{,} \PY{l+s+s2}{\PYZdq{}}\PY{l+s+s2}{y1}\PY{l+s+s2}{\PYZdq{}}\PY{p}{:} \PY{p}{[}\PY{l+m+mf}{0.35}\PY{p}{,} \PY{l+m+mf}{0.209}\PY{p}{,} \PY{l+m+mf}{0.13}\PY{p}{,} \PY{l+m+mf}{0.06}\PY{p}{]}\PY{p}{,} \PY{l+s+s2}{\PYZdq{}}\PY{l+s+s2}{delta\PYZus{}y1}\PY{l+s+s2}{\PYZdq{}}\PY{p}{:} \PY{p}{[}\PY{l+m+mf}{0.000625}\PY{p}{,} \PY{l+m+mf}{0.000625}\PY{p}{,} \PY{l+m+mf}{0.00041}\PY{p}{,} \PY{l+m+mf}{0.0005}\PY{p}{]}\PY{p}{\PYZcb{}}
\PY{n}{exp3\PYZus{}data} \PY{o}{=} \PY{n}{pd}\PY{o}{.}\PY{n}{DataFrame}\PY{p}{(}\PY{n}{data\PYZus{}dict}\PY{p}{)}
\end{Verbatim}
\end{tcolorbox}

    \begin{tcolorbox}[breakable, size=fbox, boxrule=1pt, pad at break*=1mm,colback=cellbackground, colframe=cellborder]
\prompt{In}{incolor}{32}{\hspace{4pt}}
\begin{Verbatim}[commandchars=\\\{\}]
\PY{n}{w} \PY{o}{=} \PY{p}{[}\PY{p}{]}
\PY{n}{z} \PY{o}{=} \PY{p}{[}\PY{p}{]}
\PY{k}{for} \PY{n}{index}\PY{p}{,} \PY{n}{row} \PY{o+ow}{in} \PY{n+nb}{list}\PY{p}{(}\PY{n}{exp3\PYZus{}data}\PY{o}{.}\PY{n}{iterrows}\PY{p}{(}\PY{p}{)}\PY{p}{)}\PY{p}{:}
    \PY{n}{w}\PY{o}{.}\PY{n}{append}\PY{p}{(}\PY{n}{log}\PY{p}{(}\PY{n}{row}\PY{p}{[}\PY{l+s+s1}{\PYZsq{}}\PY{l+s+s1}{a}\PY{l+s+s1}{\PYZsq{}}\PY{p}{]}\PY{p}{)}\PY{p}{)}
    \PY{n}{z}\PY{o}{.}\PY{n}{append}\PY{p}{(}\PY{n}{log}\PY{p}{(}\PY{n}{row}\PY{p}{[}\PY{l+s+s1}{\PYZsq{}}\PY{l+s+s1}{y1}\PY{l+s+s1}{\PYZsq{}}\PY{p}{]}\PY{p}{)}\PY{p}{)}
\PY{n}{wz} \PY{o}{=} \PY{n}{pd}\PY{o}{.}\PY{n}{DataFrame}\PY{p}{(}\PY{p}{\PYZob{}}\PY{l+s+s1}{\PYZsq{}}\PY{l+s+s1}{w}\PY{l+s+s1}{\PYZsq{}}\PY{p}{:} \PY{n}{w}\PY{p}{,} \PY{l+s+s1}{\PYZsq{}}\PY{l+s+s1}{z}\PY{l+s+s1}{\PYZsq{}}\PY{p}{:} \PY{n}{z}\PY{p}{\PYZcb{}}\PY{p}{)}
\PY{n}{wz}\PY{p}{[}\PY{l+s+s1}{\PYZsq{}}\PY{l+s+s1}{dz}\PY{l+s+s1}{\PYZsq{}}\PY{p}{]} \PY{o}{=} \PY{n}{ex}\PY{o}{.}\PY{n}{delta\PYZus{}y1} \PY{o}{/} \PY{n}{wz}\PY{o}{.}\PY{n}{z}
\end{Verbatim}
\end{tcolorbox}

    \begin{tcolorbox}[breakable, size=fbox, boxrule=1pt, pad at break*=1mm,colback=cellbackground, colframe=cellborder]
\prompt{In}{incolor}{16}{\hspace{4pt}}
\begin{Verbatim}[commandchars=\\\{\}]
\PY{n}{data} \PY{o}{=} \PY{p}{\PYZob{}}\PY{l+s+s1}{\PYZsq{}}\PY{l+s+s1}{n}\PY{l+s+s1}{\PYZsq{}}\PY{p}{:}\PY{p}{[}\PY{l+m+mi}{3}\PY{p}{,} \PY{l+m+mi}{4}\PY{p}{,} \PY{l+m+mi}{10}\PY{p}{,} \PY{l+m+mi}{20}\PY{p}{]}\PY{p}{,}
        \PY{l+s+s1}{\PYZsq{}}\PY{l+s+s1}{a}\PY{l+s+s1}{\PYZsq{}}\PY{p}{:}\PY{p}{[}\PY{l+m+mi}{1}\PY{p}{,} \PY{l+m+mi}{1}\PY{p}{,} \PY{l+m+mi}{1}\PY{p}{,} \PY{l+m+mi}{1}\PY{p}{]}\PY{p}{,}
        \PY{l+s+s1}{\PYZsq{}}\PY{l+s+s1}{d}\PY{l+s+s1}{\PYZsq{}}\PY{p}{:}\PY{p}{[}\PY{l+m+mi}{2}\PY{p}{,} \PY{l+m+mi}{2}\PY{p}{,} \PY{l+m+mi}{2}\PY{p}{,} \PY{l+m+mi}{2}\PY{p}{]}\PY{p}{,}
       \PY{l+s+s1}{\PYZsq{}}\PY{l+s+s1}{w}\PY{l+s+s1}{\PYZsq{}}\PY{p}{:}\PY{p}{[}\PY{l+m+mf}{3.46}\PY{p}{,} \PY{l+m+mf}{3.48}\PY{p}{,} \PY{l+m+mf}{3.51}\PY{p}{,} \PY{l+m+mf}{3.52}\PY{p}{]}\PY{p}{,}
       \PY{l+s+s1}{\PYZsq{}}\PY{l+s+s1}{dw}\PY{l+s+s1}{\PYZsq{}}\PY{p}{:}\PY{p}{[}\PY{o}{.}\PY{l+m+mi}{005}\PY{p}{,} \PY{o}{.}\PY{l+m+mi}{005}\PY{p}{,} \PY{o}{.}\PY{l+m+mi}{005}\PY{p}{,} \PY{o}{.}\PY{l+m+mi}{005}\PY{p}{]}\PY{p}{\PYZcb{}}
\PY{n}{df} \PY{o}{=} \PY{n}{pd}\PY{o}{.}\PY{n}{DataFrame}\PY{p}{(}\PY{n}{data}\PY{p}{)}
\end{Verbatim}
\end{tcolorbox}

    \begin{tcolorbox}[breakable, size=fbox, boxrule=1pt, pad at break*=1mm,colback=cellbackground, colframe=cellborder]
\prompt{In}{incolor}{10}{\hspace{4pt}}
\begin{Verbatim}[commandchars=\\\{\}]
\PY{n}{df}\PY{p}{[}\PY{l+s+s1}{\PYZsq{}}\PY{l+s+s1}{nw}\PY{l+s+s1}{\PYZsq{}}\PY{p}{]} \PY{o}{=} \PY{n}{df}\PY{p}{[}\PY{l+s+s1}{\PYZsq{}}\PY{l+s+s1}{n}\PY{l+s+s1}{\PYZsq{}}\PY{p}{]} \PY{o}{*} \PY{n}{df}\PY{p}{[}\PY{l+s+s1}{\PYZsq{}}\PY{l+s+s1}{w}\PY{l+s+s1}{\PYZsq{}}\PY{p}{]}
\PY{n}{df}\PY{p}{[}\PY{l+s+s1}{\PYZsq{}}\PY{l+s+s1}{n2}\PY{l+s+s1}{\PYZsq{}}\PY{p}{]} \PY{o}{=} \PY{n}{df}\PY{p}{[}\PY{l+s+s1}{\PYZsq{}}\PY{l+s+s1}{n}\PY{l+s+s1}{\PYZsq{}}\PY{p}{]}\PY{o}{*}\PY{o}{*}\PY{l+m+mi}{2}

\PY{n}{nwmean} \PY{o}{=} \PY{n}{np}\PY{o}{.}\PY{n}{mean}\PY{p}{(}\PY{n}{df}\PY{p}{[}\PY{l+s+s1}{\PYZsq{}}\PY{l+s+s1}{nw}\PY{l+s+s1}{\PYZsq{}}\PY{p}{]}\PY{p}{)}
\PY{n}{n2mean} \PY{o}{=} \PY{n}{np}\PY{o}{.}\PY{n}{mean}\PY{p}{(}\PY{n}{df}\PY{p}{[}\PY{l+s+s1}{\PYZsq{}}\PY{l+s+s1}{n2}\PY{l+s+s1}{\PYZsq{}}\PY{p}{]}\PY{p}{)}
\PY{n}{nmean} \PY{o}{=} \PY{n}{np}\PY{o}{.}\PY{n}{mean}\PY{p}{(}\PY{n}{df}\PY{p}{[}\PY{l+s+s1}{\PYZsq{}}\PY{l+s+s1}{n}\PY{l+s+s1}{\PYZsq{}}\PY{p}{]}\PY{p}{)}
\PY{n}{wmean} \PY{o}{=} \PY{n}{np}\PY{o}{.}\PY{n}{mean}\PY{p}{(}\PY{n}{df}\PY{p}{[}\PY{l+s+s1}{\PYZsq{}}\PY{l+s+s1}{w}\PY{l+s+s1}{\PYZsq{}}\PY{p}{]}\PY{p}{)}
\PY{n}{nmean2} \PY{o}{=} \PY{n}{nmean} \PY{o}{*}\PY{o}{*}\PY{l+m+mi}{2}
\PY{n}{sigman2} \PY{o}{=} \PY{n}{n2mean} \PY{o}{\PYZhy{}} \PY{n}{nmean2}
\end{Verbatim}
\end{tcolorbox}

    The following script outputs the values of \(\hat{m}\) and \(\hat{b}\).

    \begin{tcolorbox}[breakable, size=fbox, boxrule=1pt, pad at break*=1mm,colback=cellbackground, colframe=cellborder]
\prompt{In}{incolor}{11}{\hspace{4pt}}
\begin{Verbatim}[commandchars=\\\{\}]
\PY{n}{mhat} \PY{o}{=} \PY{p}{(}\PY{n}{nwmean} \PY{o}{\PYZhy{}} \PY{n}{nmean} \PY{o}{*} \PY{n}{wmean}\PY{p}{)}\PY{o}{/}\PY{n}{sigman2}
\PY{n}{bhat} \PY{o}{=} \PY{p}{(}\PY{n}{n2mean} \PY{o}{*} \PY{n}{wmean} \PY{o}{\PYZhy{}} \PY{n}{nmean} \PY{o}{*} \PY{n}{nwmean}\PY{p}{)}\PY{o}{/}\PY{n}{sigman2}
\PY{n+nb}{print}\PY{p}{(}\PY{n}{mhat}\PY{p}{)}
\PY{n+nb}{print}\PY{p}{(}\PY{n}{bhat}\PY{p}{)}
\end{Verbatim}
\end{tcolorbox}

    \begin{Verbatim}[commandchars=\\\{\}]
0.0031600547195623245
3.4632694938440474
\end{Verbatim}

    The following script outputs the uncertainties in the above quantities:

    \begin{tcolorbox}[breakable, size=fbox, boxrule=1pt, pad at break*=1mm,colback=cellbackground, colframe=cellborder]
\prompt{In}{incolor}{22}{\hspace{4pt}}
\begin{Verbatim}[commandchars=\\\{\}]
\PY{n}{df}\PY{p}{[}\PY{l+s+s1}{\PYZsq{}}\PY{l+s+s1}{expw}\PY{l+s+s1}{\PYZsq{}}\PY{p}{]} \PY{o}{=} \PY{n}{df}\PY{p}{[}\PY{l+s+s1}{\PYZsq{}}\PY{l+s+s1}{n}\PY{l+s+s1}{\PYZsq{}}\PY{p}{]} \PY{o}{*} \PY{n}{mhat} \PY{o}{+} \PY{n}{bhat}
\PY{n}{delw} \PY{o}{=} \PY{n}{np}\PY{o}{.}\PY{n}{sqrt}\PY{p}{(}\PY{n}{np}\PY{o}{.}\PY{n}{sum}\PY{p}{(}\PY{p}{(}\PY{n}{df}\PY{p}{[}\PY{l+s+s1}{\PYZsq{}}\PY{l+s+s1}{w}\PY{l+s+s1}{\PYZsq{}}\PY{p}{]} \PY{o}{\PYZhy{}} \PY{n}{df}\PY{p}{[}\PY{l+s+s1}{\PYZsq{}}\PY{l+s+s1}{expw}\PY{l+s+s1}{\PYZsq{}}\PY{p}{]}\PY{p}{)}\PY{o}{*}\PY{o}{*}\PY{l+m+mi}{2}\PY{p}{)}\PY{o}{/}\PY{l+m+mi}{3}\PY{p}{)}
\PY{n}{delmhat} \PY{o}{=} \PY{n}{delw}\PY{o}{/}\PY{n}{np}\PY{o}{.}\PY{n}{sqrt}\PY{p}{(}\PY{l+m+mi}{4} \PY{o}{*} \PY{n}{sigman2}\PY{p}{)}
\PY{n}{delbhat} \PY{o}{=} \PY{n}{delmhat} \PY{o}{*} \PY{n}{np}\PY{o}{.}\PY{n}{sqrt}\PY{p}{(}\PY{n}{n2mean}\PY{p}{)}
\PY{n+nb}{print}\PY{p}{(}\PY{n}{delw}\PY{p}{)}
\PY{n+nb}{print}\PY{p}{(}\PY{n}{delmhat}\PY{p}{)}
\PY{n+nb}{print}\PY{p}{(}\PY{n}{delbhat}\PY{p}{)}
\end{Verbatim}
\end{tcolorbox}

    \begin{Verbatim}[commandchars=\\\{\}]
0.012248379477204379
0.0009060453593812888
0.010380053606070557
\end{Verbatim}

    \begin{tcolorbox}[breakable, size=fbox, boxrule=1pt, pad at break*=1mm,colback=cellbackground, colframe=cellborder]
\prompt{In}{incolor}{23}{\hspace{4pt}}
\begin{Verbatim}[commandchars=\\\{\}]
\PY{n}{df}\PY{p}{[}\PY{l+s+s1}{\PYZsq{}}\PY{l+s+s1}{residual}\PY{l+s+s1}{\PYZsq{}}\PY{p}{]} \PY{o}{=} \PY{n}{df}\PY{p}{[}\PY{l+s+s1}{\PYZsq{}}\PY{l+s+s1}{w}\PY{l+s+s1}{\PYZsq{}}\PY{p}{]} \PY{o}{\PYZhy{}} \PY{n}{df}\PY{p}{[}\PY{l+s+s1}{\PYZsq{}}\PY{l+s+s1}{expw}\PY{l+s+s1}{\PYZsq{}}\PY{p}{]}
\PY{n}{df}\PY{p}{[}\PY{l+s+s1}{\PYZsq{}}\PY{l+s+s1}{delresid}\PY{l+s+s1}{\PYZsq{}}\PY{p}{]} \PY{o}{=} \PY{n}{df}\PY{p}{[}\PY{l+s+s1}{\PYZsq{}}\PY{l+s+s1}{dw}\PY{l+s+s1}{\PYZsq{}}\PY{p}{]} \PY{o}{+} \PY{n}{delw}
\end{Verbatim}
\end{tcolorbox}


    % Add a bibliography block to the postdoc
    
    
    
    \end{document}
